% Options for packages loaded elsewhere
\PassOptionsToPackage{unicode}{hyperref}
\PassOptionsToPackage{hyphens}{url}
\documentclass[
]{article}
\usepackage{xcolor}
\usepackage[margin=1in]{geometry}
\usepackage{amsmath,amssymb}
\setcounter{secnumdepth}{-\maxdimen} % remove section numbering
\usepackage{iftex}
\ifPDFTeX
  \usepackage[T1]{fontenc}
  \usepackage[utf8]{inputenc}
  \usepackage{textcomp} % provide euro and other symbols
\else % if luatex or xetex
  \usepackage{unicode-math} % this also loads fontspec
  \defaultfontfeatures{Scale=MatchLowercase}
  \defaultfontfeatures[\rmfamily]{Ligatures=TeX,Scale=1}
\fi
\usepackage{lmodern}
\ifPDFTeX\else
  % xetex/luatex font selection
\fi
% Use upquote if available, for straight quotes in verbatim environments
\IfFileExists{upquote.sty}{\usepackage{upquote}}{}
\IfFileExists{microtype.sty}{% use microtype if available
  \usepackage[]{microtype}
  \UseMicrotypeSet[protrusion]{basicmath} % disable protrusion for tt fonts
}{}
\makeatletter
\@ifundefined{KOMAClassName}{% if non-KOMA class
  \IfFileExists{parskip.sty}{%
    \usepackage{parskip}
  }{% else
    \setlength{\parindent}{0pt}
    \setlength{\parskip}{6pt plus 2pt minus 1pt}}
}{% if KOMA class
  \KOMAoptions{parskip=half}}
\makeatother
\usepackage{color}
\usepackage{fancyvrb}
\newcommand{\VerbBar}{|}
\newcommand{\VERB}{\Verb[commandchars=\\\{\}]}
\DefineVerbatimEnvironment{Highlighting}{Verbatim}{commandchars=\\\{\}}
% Add ',fontsize=\small' for more characters per line
\usepackage{framed}
\definecolor{shadecolor}{RGB}{248,248,248}
\newenvironment{Shaded}{\begin{snugshade}}{\end{snugshade}}
\newcommand{\AlertTok}[1]{\textcolor[rgb]{0.94,0.16,0.16}{#1}}
\newcommand{\AnnotationTok}[1]{\textcolor[rgb]{0.56,0.35,0.01}{\textbf{\textit{#1}}}}
\newcommand{\AttributeTok}[1]{\textcolor[rgb]{0.13,0.29,0.53}{#1}}
\newcommand{\BaseNTok}[1]{\textcolor[rgb]{0.00,0.00,0.81}{#1}}
\newcommand{\BuiltInTok}[1]{#1}
\newcommand{\CharTok}[1]{\textcolor[rgb]{0.31,0.60,0.02}{#1}}
\newcommand{\CommentTok}[1]{\textcolor[rgb]{0.56,0.35,0.01}{\textit{#1}}}
\newcommand{\CommentVarTok}[1]{\textcolor[rgb]{0.56,0.35,0.01}{\textbf{\textit{#1}}}}
\newcommand{\ConstantTok}[1]{\textcolor[rgb]{0.56,0.35,0.01}{#1}}
\newcommand{\ControlFlowTok}[1]{\textcolor[rgb]{0.13,0.29,0.53}{\textbf{#1}}}
\newcommand{\DataTypeTok}[1]{\textcolor[rgb]{0.13,0.29,0.53}{#1}}
\newcommand{\DecValTok}[1]{\textcolor[rgb]{0.00,0.00,0.81}{#1}}
\newcommand{\DocumentationTok}[1]{\textcolor[rgb]{0.56,0.35,0.01}{\textbf{\textit{#1}}}}
\newcommand{\ErrorTok}[1]{\textcolor[rgb]{0.64,0.00,0.00}{\textbf{#1}}}
\newcommand{\ExtensionTok}[1]{#1}
\newcommand{\FloatTok}[1]{\textcolor[rgb]{0.00,0.00,0.81}{#1}}
\newcommand{\FunctionTok}[1]{\textcolor[rgb]{0.13,0.29,0.53}{\textbf{#1}}}
\newcommand{\ImportTok}[1]{#1}
\newcommand{\InformationTok}[1]{\textcolor[rgb]{0.56,0.35,0.01}{\textbf{\textit{#1}}}}
\newcommand{\KeywordTok}[1]{\textcolor[rgb]{0.13,0.29,0.53}{\textbf{#1}}}
\newcommand{\NormalTok}[1]{#1}
\newcommand{\OperatorTok}[1]{\textcolor[rgb]{0.81,0.36,0.00}{\textbf{#1}}}
\newcommand{\OtherTok}[1]{\textcolor[rgb]{0.56,0.35,0.01}{#1}}
\newcommand{\PreprocessorTok}[1]{\textcolor[rgb]{0.56,0.35,0.01}{\textit{#1}}}
\newcommand{\RegionMarkerTok}[1]{#1}
\newcommand{\SpecialCharTok}[1]{\textcolor[rgb]{0.81,0.36,0.00}{\textbf{#1}}}
\newcommand{\SpecialStringTok}[1]{\textcolor[rgb]{0.31,0.60,0.02}{#1}}
\newcommand{\StringTok}[1]{\textcolor[rgb]{0.31,0.60,0.02}{#1}}
\newcommand{\VariableTok}[1]{\textcolor[rgb]{0.00,0.00,0.00}{#1}}
\newcommand{\VerbatimStringTok}[1]{\textcolor[rgb]{0.31,0.60,0.02}{#1}}
\newcommand{\WarningTok}[1]{\textcolor[rgb]{0.56,0.35,0.01}{\textbf{\textit{#1}}}}
\usepackage{graphicx}
\makeatletter
\newsavebox\pandoc@box
\newcommand*\pandocbounded[1]{% scales image to fit in text height/width
  \sbox\pandoc@box{#1}%
  \Gscale@div\@tempa{\textheight}{\dimexpr\ht\pandoc@box+\dp\pandoc@box\relax}%
  \Gscale@div\@tempb{\linewidth}{\wd\pandoc@box}%
  \ifdim\@tempb\p@<\@tempa\p@\let\@tempa\@tempb\fi% select the smaller of both
  \ifdim\@tempa\p@<\p@\scalebox{\@tempa}{\usebox\pandoc@box}%
  \else\usebox{\pandoc@box}%
  \fi%
}
% Set default figure placement to htbp
\def\fps@figure{htbp}
\makeatother
\setlength{\emergencystretch}{3em} % prevent overfull lines
\providecommand{\tightlist}{%
  \setlength{\itemsep}{0pt}\setlength{\parskip}{0pt}}
\usepackage{bookmark}
\IfFileExists{xurl.sty}{\usepackage{xurl}}{} % add URL line breaks if available
\urlstyle{same}
\hypersetup{
  pdftitle={Midterm 1 W26},
  pdfauthor={Arden Moser},
  hidelinks,
  pdfcreator={LaTeX via pandoc}}

\title{Midterm 1 W26}
\author{Arden Moser}
\date{2026-01-29}

\begin{document}
\maketitle

\subsection{Instructions}\label{instructions}

Answer the following questions and complete the exercises in RMarkdown.
Please embed all of your code and push your final work to your
repository. Your code must be organized, clean, and run free from
errors. Remember, you must remove the \texttt{\#} for any included code
chunks to run. Be sure to add your name to the author header above.

Your code must knit in order to be considered. If you are stuck and
cannot answer a question, then comment out your code and knit the
document. You may use your notes, labs, and homework to help you
complete this exam. Do not use any other resources- including AI
assistance or other students' work.

Don't forget to answer any questions that are asked in the prompt! Each
question must be coded; it cannot be answered by a sort in a spreadsheet
or a written response only.

For all plots you create, a title and clearly labeled axes must be
provided. We also expect pipes \texttt{\%\textgreater{}\%} to be used
wherever possible.

Be sure to push your completed midterm to your repository and upload the
document to Gradescope. This exam is worth 50 points.

Please load the following libraries.

\begin{Shaded}
\begin{Highlighting}[]
\FunctionTok{library}\NormalTok{(tidyverse)}
\FunctionTok{library}\NormalTok{(janitor)}
\end{Highlighting}
\end{Shaded}

\subsection{Part 1: Repository}\label{part-1-repository}

\textbf{Question 1. (3 points) Before you start analyzing data, please
put a link to your GitHub repository below. Your repository should have
a clear README and be well-organized. Add \texttt{jmledford3115} and
\texttt{bryshalm} as collaborators to your repository if you haven't
already done so.}

Link to repository:
\href{https://github.com/ammoser-gif/BIS15L_W26_amoser.git}{My
Repository}

\subsection{Part 2: Data and Analysis}\label{part-2-data-and-analysis}

In the midterm 1 folder there is a second folder called \texttt{data}.
Inside the \texttt{data} folder, there is a .csv file called
\texttt{anolis\_dat.csv}. These data came from
\href{https://academic.oup.com/evolut/article/64/9/2731/6854302?login=true}{D.
Luke Mahler, Liam J. Revell, Richard E. Glor, Jonathan B. Losos,
ECOLOGICAL OPPORTUNITY AND THE RATE OF MORPHOLOGICAL EVOLUTION IN THE
DIVERSIFICATION OF GREATER ANTILLEAN ANOLES, Evolution, Volume 64, Issue
9, 1 September 2010, Pages 2731--2745}. The original research article is
included in the data folder.

\emph{Anolis} is a genus of lizards commonly known as anoles. Anoles are
found throughout the Americas, but are especially diverse in the
Caribbean. The data include morphological measurements for \emph{Anolis}
lizards from the islands of the Greater Antilles. These data can be used
to study patterns of morphological evolution and adaptation in
\emph{Anolis} lizards.

The variables include:\\
- \texttt{species}: Species name of the anole lizard.\\
- \texttt{habitat}: Habitat type where the lizard was found.\\
- \texttt{hindlimb\_length\_mm}: Length of the lizard's hindlimbs (in
millimeters).\\
- \texttt{tail\_length\_mm}: Length of the lizard's tail (in
millimeters).\\
- \texttt{body\_length\_mm}: Length of the lizard's body (in
millimeters).\\
- \texttt{toepad\_lamellae\_count}: Count of lamellae on the lizard's
toepads.\\
- \texttt{island}: Island where the lizard was found.

\textbf{Question 2. (2 points) Load the data and store it as an object
called \texttt{anolis}.}

\begin{Shaded}
\begin{Highlighting}[]
\NormalTok{anolis }\OtherTok{\textless{}{-}} \FunctionTok{read\_csv}\NormalTok{(}\StringTok{"data/anolis\_dat.csv"}\NormalTok{)}
\end{Highlighting}
\end{Shaded}

\begin{verbatim}
## Rows: 52 Columns: 7
## -- Column specification --------------------------------------------------------
## Delimiter: ","
## chr (3): Species, Habitat, Island
## dbl (4): Hindlimb length (mm), Tail length (mm), Body length (mm), Toepad la...
## 
## i Use `spec()` to retrieve the full column specification for this data.
## i Specify the column types or set `show_col_types = FALSE` to quiet this message.
\end{verbatim}

\hfill\break
\textbf{Question 3. (2 points) Use a summary function of your choice to
get an idea of the structure of the data.}

\begin{Shaded}
\begin{Highlighting}[]
\FunctionTok{glimpse}\NormalTok{(anolis)}
\end{Highlighting}
\end{Shaded}

\begin{verbatim}
## Rows: 52
## Columns: 7
## $ Species                   <chr> "A. ahli", "A. alayoni", "A. alfaroi", "A. a~
## $ Habitat                   <chr> "Trunk-ground", "Twig", "Grass-bush", "Trunk~
## $ `Hindlimb length (mm)`    <dbl> 50.46, 25.50, 26.17, 36.80, 50.39, 49.37, 29~
## $ `Tail length (mm)`        <dbl> 81.99, 54.75, 79.00, 84.88, 154.45, 91.01, 1~
## $ `Body length (mm)`        <dbl> 51.67, 41.32, 30.95, 51.53, 72.32, 51.72, 32~
## $ `Toepad lamellae (count)` <dbl> 27, 31, 24, 36, 41, 28, 29, 28, 28, 31, 32, ~
## $ Island                    <chr> "Cuba", "Cuba", "Cuba", "Hispaniola", "Cuba"~
\end{verbatim}

\hfill\break
\textbf{Question 4. (2 points) Clean the variable names so they are all
lowercase and without special characters or spaces. Be sure to use the
cleaned data for all subsequent analyses.}

\begin{Shaded}
\begin{Highlighting}[]
\NormalTok{anolis\_cleaned }\OtherTok{\textless{}{-}}\NormalTok{ anolis }\SpecialCharTok{\%\textgreater{}\%} \FunctionTok{clean\_names}\NormalTok{()}
\NormalTok{anolis\_cleaned}
\end{Highlighting}
\end{Shaded}

\begin{verbatim}
## # A tibble: 52 x 7
##    species        habitat      hindlimb_length_mm tail_length_mm body_length_mm
##    <chr>          <chr>                     <dbl>          <dbl>          <dbl>
##  1 A. ahli        Trunk-ground               50.5           82.0           51.7
##  2 A. alayoni     Twig                       25.5           54.8           41.3
##  3 A. alfaroi     Grass-bush                 26.2           79             31.0
##  4 A. aliniger    Trunk-crown                36.8           84.9           51.5
##  5 A. allisoni    Trunk-crown                50.4          154.            72.3
##  6 A. allogus     Trunk-ground               49.4           91.0           51.7
##  7 A. alumina     Grass-bush                 30.0          106.            32.9
##  8 A. alutaceus   Grass-bush                 27.4           94.6           31.8
##  9 A. angusticeps Twig                       24.4           65.1           40.2
## 10 A. armouri     Trunk-ground               51.8          101.            56.1
## # i 42 more rows
## # i 2 more variables: toepad_lamellae_count <dbl>, island <chr>
\end{verbatim}

\hfill\break
\textbf{Question 5. (4 points) Convert the \texttt{habitat} and
\texttt{island} variables to factors.}

\begin{Shaded}
\begin{Highlighting}[]
\NormalTok{anolis\_cleaned\_factored }\OtherTok{\textless{}{-}}\NormalTok{ anolis\_cleaned }\SpecialCharTok{\%\textgreater{}\%} 
  \FunctionTok{mutate}\NormalTok{(}\FunctionTok{across}\NormalTok{(}\FunctionTok{c}\NormalTok{(habitat, island), as.factor))}
\NormalTok{anolis\_cleaned\_factored}
\end{Highlighting}
\end{Shaded}

\begin{verbatim}
## # A tibble: 52 x 7
##    species        habitat      hindlimb_length_mm tail_length_mm body_length_mm
##    <chr>          <fct>                     <dbl>          <dbl>          <dbl>
##  1 A. ahli        Trunk-ground               50.5           82.0           51.7
##  2 A. alayoni     Twig                       25.5           54.8           41.3
##  3 A. alfaroi     Grass-bush                 26.2           79             31.0
##  4 A. aliniger    Trunk-crown                36.8           84.9           51.5
##  5 A. allisoni    Trunk-crown                50.4          154.            72.3
##  6 A. allogus     Trunk-ground               49.4           91.0           51.7
##  7 A. alumina     Grass-bush                 30.0          106.            32.9
##  8 A. alutaceus   Grass-bush                 27.4           94.6           31.8
##  9 A. angusticeps Twig                       24.4           65.1           40.2
## 10 A. armouri     Trunk-ground               51.8          101.            56.1
## # i 42 more rows
## # i 2 more variables: toepad_lamellae_count <dbl>, island <fct>
\end{verbatim}

\hfill\break
\textbf{Question 6. (2 points) Anole species were sampled from multiple
islands. Which islands are represented in the data? Display the island
names.}

\begin{Shaded}
\begin{Highlighting}[]
\NormalTok{anolis\_cleaned }\SpecialCharTok{\%\textgreater{}\%} 
  \FunctionTok{distinct}\NormalTok{(island)}
\end{Highlighting}
\end{Shaded}

\begin{verbatim}
## # A tibble: 4 x 1
##   island     
##   <chr>      
## 1 Cuba       
## 2 Hispaniola 
## 3 Puerto Rico
## 4 Jamacia
\end{verbatim}

\hfill\break
\textbf{Cuba, Hispaniola, Puerto Rico, and Jamacia} are islands
represented in the data.\\
\strut \\
\textbf{Question 7. (4 points) Is sampling equal across islands? Create
a plot to visualize the number of anole species sampled from each
island. Be sure to label your axes and add a title.}

\begin{Shaded}
\begin{Highlighting}[]
\NormalTok{anolis\_cleaned }\SpecialCharTok{\%\textgreater{}\%} 
  \FunctionTok{ggplot}\NormalTok{(}\FunctionTok{aes}\NormalTok{(}\AttributeTok{x =}\NormalTok{ island, }\AttributeTok{fill =}\NormalTok{ island)) }\SpecialCharTok{+}
  \FunctionTok{geom\_bar}\NormalTok{() }\SpecialCharTok{+} 
  \FunctionTok{labs}\NormalTok{(}
    \AttributeTok{title =} \StringTok{"Number of Anole Species per Island"}\NormalTok{,}
    \AttributeTok{x =} \StringTok{"Island"}\NormalTok{,}
    \AttributeTok{y =} \StringTok{"Number of Anole Species"}
\NormalTok{  )}
\end{Highlighting}
\end{Shaded}

\pandocbounded{\includegraphics[keepaspectratio]{midterm1_files/figure-latex/unnamed-chunk-7-1.pdf}}\\
Sampling is not equal across the islands because Cuba has significantly
more anole species recorded than the rest of the islands. For instance,
Cuba expresses more than 20 species compared to Jamacia's count of under
5 species.\\

\textbf{Question 8. (2 points) Which habitat types are represented in
the data? Display the names of the habitat types.}

\begin{Shaded}
\begin{Highlighting}[]
\NormalTok{anolis\_cleaned }\SpecialCharTok{\%\textgreater{}\%} 
  \FunctionTok{distinct}\NormalTok{(habitat)}
\end{Highlighting}
\end{Shaded}

\begin{verbatim}
## # A tibble: 4 x 1
##   habitat     
##   <chr>       
## 1 Trunk-ground
## 2 Twig        
## 3 Grass-bush  
## 4 Trunk-crown
\end{verbatim}

\hfill\break
\textbf{Trunk-ground, twig, grass-bush, and trunk-crown} are habitat
types represented in the data.\\

\textbf{Question 9. (4 points) Is sampling equal across habitat types?
Create a plot to visualize the number of anole species sampled from each
habitat type. Be sure to label your axes and add a title.}

\begin{Shaded}
\begin{Highlighting}[]
\NormalTok{anolis\_cleaned }\SpecialCharTok{\%\textgreater{}\%} 
  \FunctionTok{ggplot}\NormalTok{(}\FunctionTok{aes}\NormalTok{(}\AttributeTok{x =}\NormalTok{ habitat, }\AttributeTok{fill =}\NormalTok{ habitat)) }\SpecialCharTok{+}
  \FunctionTok{geom\_bar}\NormalTok{() }\SpecialCharTok{+} 
  \FunctionTok{labs}\NormalTok{(}
    \AttributeTok{title =} \StringTok{"Number of Anole Species across Habitat Types"}\NormalTok{,}
    \AttributeTok{x =} \StringTok{"Habitat Type"}\NormalTok{,}
    \AttributeTok{y =} \StringTok{"Number of Anole Species"}
\NormalTok{  )}
\end{Highlighting}
\end{Shaded}

\pandocbounded{\includegraphics[keepaspectratio]{midterm1_files/figure-latex/unnamed-chunk-9-1.pdf}}\\
Sampling is not equal across habitat types because Twig has
significantly less anole species recorded than the rest of the habitats.
For instance, Twig expresses a mere 5 species compared to Trunk-ground's
count of over 20 species.\\

\textbf{Question 10. (4 points) The morphology of anoles varies based on
their habitat. How does the range of hindlimb length compare among
different habitats? Create a plot to visualize the distribution of
hindlimb lengths across habitat types. Be sure to label your axes and
add a title.}

\begin{Shaded}
\begin{Highlighting}[]
\NormalTok{anolis\_cleaned }\SpecialCharTok{\%\textgreater{}\%} 
  \FunctionTok{ggplot}\NormalTok{(}\FunctionTok{aes}\NormalTok{(}\AttributeTok{x =}\NormalTok{ hindlimb\_length\_mm, }\AttributeTok{y =}\NormalTok{ habitat, }\AttributeTok{fill =}\NormalTok{ habitat)) }\SpecialCharTok{+}
  \FunctionTok{geom\_boxplot}\NormalTok{() }\SpecialCharTok{+}
  \FunctionTok{labs}\NormalTok{(}
    \AttributeTok{title =} \StringTok{"Hind Limb Length of Anolis Species across Different Habitats"}\NormalTok{,}
    \AttributeTok{x =} \StringTok{"Hind Limb Length (mm)"}\NormalTok{,}
    \AttributeTok{y =} \StringTok{"Habitat Type"}
\NormalTok{  )}
\end{Highlighting}
\end{Shaded}

\pandocbounded{\includegraphics[keepaspectratio]{midterm1_files/figure-latex/unnamed-chunk-10-1.pdf}}\\
From widest range to narrowest range of hind limb lengths, the habitat
types are ordered as follows: Trunk-crown, Grass-bush, Trunk-ground, and
Twig. Trunk-ground's range overlaps with Trunk-crown's range.\\

\textbf{Question 11. (4 points) The plot above is compelling, but don't
we expect larger lizards to have longer limbs? What about tail length?
Shouldn't longer lizards have longer tails? To correct for this, make
two new columns: 1. \texttt{ratio\_of\_hindlimb\_to\_body}, and 2.
\texttt{ratio\_of\_tail\_to\_body}. Don't forget to add these columns to
the anolis data frame.}

\begin{Shaded}
\begin{Highlighting}[]
\NormalTok{anolis\_cleaned }\SpecialCharTok{\%\textgreater{}\%} 
  \FunctionTok{mutate}\NormalTok{(}\AttributeTok{ratio\_of\_hindlimb\_to\_body =}\NormalTok{ hindlimb\_length\_mm }\SpecialCharTok{/}\NormalTok{ body\_length\_mm) }\SpecialCharTok{\%\textgreater{}\%} 
  \FunctionTok{mutate}\NormalTok{(}\AttributeTok{ratio\_of\_tail\_to\_body =}\NormalTok{ tail\_length\_mm }\SpecialCharTok{/}\NormalTok{ body\_length\_mm)}
\end{Highlighting}
\end{Shaded}

\begin{verbatim}
## # A tibble: 52 x 9
##    species        habitat      hindlimb_length_mm tail_length_mm body_length_mm
##    <chr>          <chr>                     <dbl>          <dbl>          <dbl>
##  1 A. ahli        Trunk-ground               50.5           82.0           51.7
##  2 A. alayoni     Twig                       25.5           54.8           41.3
##  3 A. alfaroi     Grass-bush                 26.2           79             31.0
##  4 A. aliniger    Trunk-crown                36.8           84.9           51.5
##  5 A. allisoni    Trunk-crown                50.4          154.            72.3
##  6 A. allogus     Trunk-ground               49.4           91.0           51.7
##  7 A. alumina     Grass-bush                 30.0          106.            32.9
##  8 A. alutaceus   Grass-bush                 27.4           94.6           31.8
##  9 A. angusticeps Twig                       24.4           65.1           40.2
## 10 A. armouri     Trunk-ground               51.8          101.            56.1
## # i 42 more rows
## # i 4 more variables: toepad_lamellae_count <dbl>, island <chr>,
## #   ratio_of_hindlimb_to_body <dbl>, ratio_of_tail_to_body <dbl>
\end{verbatim}

\hfill\break

\textbf{Question 12. (4 points) Create a new plot that examines the
distribution of \texttt{ratio\_of\_hindlimb\_to\_body} across habitat
types. How does this plot differ from the one you made in Problem 10? Be
sure to label your axes and add a title.}

\begin{Shaded}
\begin{Highlighting}[]
\NormalTok{anolis\_cleaned }\SpecialCharTok{\%\textgreater{}\%} 
  \FunctionTok{mutate}\NormalTok{(}\AttributeTok{ratio\_of\_hindlimb\_to\_body =}\NormalTok{ hindlimb\_length\_mm }\SpecialCharTok{/}\NormalTok{ body\_length\_mm) }\SpecialCharTok{\%\textgreater{}\%} 
  \FunctionTok{ggplot}\NormalTok{(}\FunctionTok{aes}\NormalTok{(}\AttributeTok{x =}\NormalTok{ ratio\_of\_hindlimb\_to\_body, }\AttributeTok{y =}\NormalTok{ habitat, }\AttributeTok{fill =}\NormalTok{ habitat)) }\SpecialCharTok{+}
  \FunctionTok{geom\_boxplot}\NormalTok{() }\SpecialCharTok{+}
  \FunctionTok{labs}\NormalTok{(}
    \AttributeTok{title =} \StringTok{"Hind Limb to Body Length Ratio of Anolis Species across Different Habitats"}\NormalTok{,}
    \AttributeTok{x =} \StringTok{"Hind Limb to Body Length Ratio (mm/mm)"}\NormalTok{,}
    \AttributeTok{y =} \StringTok{"Habitat Type"}
\NormalTok{  )}
\end{Highlighting}
\end{Shaded}

\pandocbounded{\includegraphics[keepaspectratio]{midterm1_files/figure-latex/unnamed-chunk-12-1.pdf}}\\
The order of widest range to narrowest range changed completely when
considering the hind limb to body length ratio instead. Said order is
now as follows: Twig, Trunk-crown, Grass-bush, and Trunk-ground. Trunk
ground's range now overlaps with Grass-bush's range instead of
Trunk-crown's range.\\

\textbf{Problem 13. (4 points) A longer tail provides better balance and
agility. Create a plot that examines the relationship between body
length and tail length. Color the points by habitat type and add a line
of best fit. What does this plot suggest about the relationship between
body length and tail length? What do you notice about lizards in the
\texttt{Grass-bush} habitat? Be sure to label your axes and add a
title.}

\begin{Shaded}
\begin{Highlighting}[]
\NormalTok{anolis\_cleaned }\SpecialCharTok{\%\textgreater{}\%} 
  \FunctionTok{ggplot}\NormalTok{(}\FunctionTok{aes}\NormalTok{(}\AttributeTok{x =}\NormalTok{ body\_length\_mm, }\AttributeTok{y =}\NormalTok{ tail\_length\_mm)) }\SpecialCharTok{+}
  \FunctionTok{geom\_point}\NormalTok{(}\FunctionTok{aes}\NormalTok{(}\AttributeTok{color =}\NormalTok{ habitat)) }\SpecialCharTok{+}
  \FunctionTok{geom\_smooth}\NormalTok{(}\AttributeTok{method =} \StringTok{"lm"}\NormalTok{) }\SpecialCharTok{+}
  \FunctionTok{labs}\NormalTok{(}
    \AttributeTok{title =} \StringTok{"Body Length vs. Tail Length of Anole Species"}\NormalTok{,}
    \AttributeTok{x =} \StringTok{"Body Length (mm)"}\NormalTok{,}
    \AttributeTok{y =} \StringTok{"Tail Length (mm)"}
\NormalTok{  )}
\end{Highlighting}
\end{Shaded}

\begin{verbatim}
## `geom_smooth()` using formula = 'y ~ x'
\end{verbatim}

\pandocbounded{\includegraphics[keepaspectratio]{midterm1_files/figure-latex/unnamed-chunk-13-1.pdf}}\\
Becuase the plot presents a positive, linear line of best fit, it
suggests that body length generally correlates to a longer tail.
However, lizards in the `Grass-bush' habitat have unusually long tails
for their short bodies.\\

\textbf{Problem 14. (4 points) Toepad lamellae are transverse,
plate-like structures found on the ventral surface of the digits. They
are a key adaptation that allows anoles to cling to and move efficiently
on smooth and vertical surfaces. What is the mean number of toepad
lamellae for each habitat type?}

\begin{Shaded}
\begin{Highlighting}[]
\NormalTok{anolis\_cleaned }\SpecialCharTok{\%\textgreater{}\%} 
  \FunctionTok{select}\NormalTok{(toepad\_lamellae\_count, habitat) }\SpecialCharTok{\%\textgreater{}\%} 
  \FunctionTok{group\_by}\NormalTok{(habitat) }\SpecialCharTok{\%\textgreater{}\%} 
  \FunctionTok{summarize}\NormalTok{(}\AttributeTok{mean\_toepad\_lamellae\_count =} \FunctionTok{mean}\NormalTok{(toepad\_lamellae\_count))}
\end{Highlighting}
\end{Shaded}

\begin{verbatim}
## # A tibble: 4 x 2
##   habitat      mean_toepad_lamellae_count
##   <chr>                             <dbl>
## 1 Grass-bush                         28.3
## 2 Trunk-crown                        38.5
## 3 Trunk-ground                       30.0
## 4 Twig                               27.6
\end{verbatim}

\hfill\break
The mean number of toepad lamellae for each habitat types is about
\textbf{28 for `Grass-bush'}, \textbf{39 for `Trunk-crown'}, \textbf{30
for `Trunk-ground'}, and \textbf{28 for `Twig'}.\\

\textbf{Problem 15. (5 points) The number of toepad lamellae is
significantly different for trunk-crown species. But, is this consistent
across all islands? Make a plot that shows the range in number of toepad
lamellae by island for trunk-crown species only. Be sure to label your
axes and add a title.}

\begin{Shaded}
\begin{Highlighting}[]
\NormalTok{anolis\_cleaned }\SpecialCharTok{\%\textgreater{}\%} 
  \FunctionTok{select}\NormalTok{(toepad\_lamellae\_count, habitat, island) }\SpecialCharTok{\%\textgreater{}\%} 
  \FunctionTok{filter}\NormalTok{(habitat }\SpecialCharTok{==} \StringTok{"Trunk{-}crown"}\NormalTok{) }\SpecialCharTok{\%\textgreater{}\%} 
  \FunctionTok{ggplot}\NormalTok{(}\FunctionTok{aes}\NormalTok{(}\AttributeTok{x =}\NormalTok{ toepad\_lamellae\_count, }\AttributeTok{y =}\NormalTok{ island, }\AttributeTok{fill =}\NormalTok{ island)) }\SpecialCharTok{+} 
  \FunctionTok{geom\_col}\NormalTok{() }\SpecialCharTok{+}
  \FunctionTok{labs}\NormalTok{(}
    \AttributeTok{title =} \StringTok{"Toepad Lamellae Count for Trunk{-}crown Species across Islands"}\NormalTok{,}
    \AttributeTok{x =} \StringTok{"Toepad Lamellae Count"}\NormalTok{,}
    \AttributeTok{y =} \StringTok{"Island"}
\NormalTok{  )}
\end{Highlighting}
\end{Shaded}

\pandocbounded{\includegraphics[keepaspectratio]{midterm1_files/figure-latex/unnamed-chunk-15-1.pdf}}\\
The number of toepad lamellae among Trunk-crown species \textbf{remains
consistent} across islands. Species in Hispaniola have the most toepad
lamellae a nearly 120, followed by Cuba with about 90, Jamacia with
about 75, and Puerto Rico with little over 30 toepad lamelle.\\

\subsection{Submit the Midterm}\label{submit-the-midterm}

\begin{enumerate}
\def\labelenumi{\arabic{enumi}.}
\tightlist
\item
  Save your work and knit the .rmd file.\\
\item
  Open the .html file and ``print'' it to a .pdf file in Google Chrome
  (not Safari).\\
\item
  Go to the class Canvas page and open Gradescope.\\
\item
  Submit your .pdf file to the midterm assignment- be sure to assign the
  pages to the correct questions.\\
\item
  Commit and push your work to your repository.
\end{enumerate}

\end{document}
